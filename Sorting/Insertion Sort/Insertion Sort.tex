\documentclass[UTF8]{ctexart}
\usepackage[margin=1in]{geometry}
\usepackage{amsmath}
\usepackage{amssymb}
\usepackage{listings}
\usepackage{xcolor}
\usepackage{graphicx}
\usepackage{hyperref}

% 代码格式设置
\lstset{
    language=C++,
    basicstyle=\ttfamily\small,
    breaklines=true,
    numbers=left,
    numberstyle=\tiny,
    keywordstyle=\color{blue},
    commentstyle=\color{gray},
    stringstyle=\color{red},
    showstringspaces=false,
    backgroundcolor=\color{lightgray!20},
    frame=single
}

\title{\textbf{插入排序 (Insertion Sort)}}
\author{}
\date{}

\begin{document}

\maketitle

\section{算法思想}
插入排序的基本思想是:维护一个已排序的序列(初始时只包含输入序列的第一个元素)。每一步从待排序序列中取出一个元素,将它插入到已排序序列的适当位置,使之仍然保持有序。

\section{示例}
初始序列:
\begin{center}
$[34,\;8,\;64,\;51,\;32,\;21]$
\end{center}

按步骤插入:
\begin{itemize}
  \item 第1次插入(将 8 插入到 34 之前):$[8,\;34,\;64,\;51,\;32,\;21]$(swap)
  \item 第2次插入(64 已在正确位置):$[8,\;34,\;64,\;51,\;32,\;21]$(no swap)
  \item 第3次插入(将 51 插入到 34 和 64 之间):$[8,\;34,\;51,\;64,\;32,\;21]$(swap)
  \item 第4次插入(将 32 插入合适位置):$[8,\;34,\;51,\;32,\;64,\;21]\to[8,\;34,\;32,\;51,\;64,\;21]\to[8,\;32,\;34,\;51,\;64,\;21]$(连续 swap)
  \item 第5次插入(将 21 插入合适位置):最终 $[8,\;21,\;32,\;34,\;51,\;64]$
\end{itemize}

上述过程展示了在第 $k$ 次迭代时,将一个元素插入到长度为 $k$ 的已排序列表中,最坏情况需要 $k$ 次比较和移位。

\section{伪代码}
\begin{lstlisting}
InsertionSort(A)
for (i = 2; i <= n; i++) {
    j = i - 1;
    while (j >= 1 && A[j] > A[j+1]) {
        Swap(A[j], A[j+1]);
        j = j - 1;
    }
}
\end{lstlisting}

\section{复杂度分析}
在第 $k$ 次迭代中,将一个元素插入到已排序的 $k$ 个元素中,最坏情况需要进行 $k$ 次比较与移动。因此,总的比较/移动次数在最坏情况为:
\[
\sum_{k=1}^{n-1} k = \frac{n(n-1)}{2} = \Theta(n^2).
\]

\subsection{最坏情况}
当输入序列按降序排列时,插入排序每次插入都需要移动到最前面,总比较/移动次数为 $\Theta(n^2)$。

\subsection{平均情况}
平均情况下比较次数仍为 $\Theta(n^2)$(可通过倒置数分析得到)。

\subsection{最好情况}
当输入序列已按非降序排列时,每次插入不需要移动,比较次数为 $n-1$,因此最好情况为 $\Theta(n)$。

\section{倒置论证(与冒泡排序类似)}
对于仅交换相邻元素的排序算法,采用倒置(inversion)概念可以得到平均下界。定义:若 $i<j$ 且 $A[i]>A[j]$,则称该对 $(i,j)$ 为一个倒置。每次交换相邻元素至多消除一个倒置,因此任何仅通过相邻交换进行排序的算法,其交换次数至少等于输入中的倒置数。

在所有 $n!$ 个排列的均匀分布下,任意一对元素在排列中构成倒置的概率为 $1/2$,因此期望倒置数为 $\binom{n}{2}/2 = n(n-1)/4$。由此可知,基于相邻交换的排序算法(例如插入排序、冒泡排序)在平均情况下需要 $\Omega(n^2)$ 次交换与比较,也因此平均时间复杂度为 $\Theta(n^2)$。

\section{总结}
插入排序实现简单、稳定,且对接近有序的数据非常高效(接近线性时间)。但在一般随机输入或最坏情况下,时间复杂度为 $\Theta(n^2)$,因而不适合大型数据集的排序任务。

\end{document}
