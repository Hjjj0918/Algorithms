\documentclass[UTF8]{ctexart}
\usepackage[margin=1in]{geometry}
\usepackage{amsmath}
\usepackage{amssymb}
\usepackage{listings}
\usepackage{xcolor}
\usepackage{graphicx}
\usepackage{hyperref}
\usepackage{algorithmic}
\usepackage{algorithm}

% 代码格式设置
\lstset{
    language=C++,
    basicstyle=\ttfamily\small,
    breaklines=true,
    numbers=left,
    numberstyle=\tiny,
    keywordstyle=\color{blue},
    commentstyle=\color{gray},
    stringstyle=\color{red},
    showstringspaces=false,
    backgroundcolor=\color{lightgray!20},
    frame=single
}

\title{\textbf{选择排序 (Selection Sort)}}
\author{}
\date{}

\begin{document}

\maketitle

\section{选择排序的基本思想}

选择排序是一种简单的排序算法,其核心思想是:

\begin{itemize}
    \item 在每一轮迭代中,找出未排序部分的最大元素
    \item 将该最大元素放到正确的最终位置(已排序部分的开始)
    \item 重复此过程,直到整个数组排序完成
\end{itemize}

与冒泡排序不同,选择排序在找到一个元素的正确位置后,不会再移动它,这是其主要特性。

\section{算法步骤示例}

以数组 $[34, 8, 64, 51, 32, 21]$ 为例,演示选择排序的过程:

\begin{enumerate}
    \item \textbf{初始数组:} $[34, 8, 64, 51, 32, 21]$
    \item \textbf{第1轮:} 找最大值 64,放在末尾
        \begin{center}
            $[34, 8, 21, 51, 32] \mid 64$
        \end{center}
    \item \textbf{第2轮:} 找最大值 51,放在倒数第二位
        \begin{center}
            $[34, 8, 21, 32] \mid 51, 64$
        \end{center}
    \item \textbf{第3轮:} 找最大值 34,放在倒数第三位
        \begin{center}
            $[32, 8, 21] \mid 34, 51, 64$
        \end{center}
    \item \textbf{第4轮:} 找最大值 32,放在倒数第四位
        \begin{center}
            $[21, 8] \mid 32, 34, 51, 64$
        \end{center}
    \item \textbf{第5轮:} 找最大值 21,放在倒数第五位
        \begin{center}
            $[8] \mid 21, 32, 34, 51, 64$
        \end{center}
    \item \textbf{最终结果:} $[8, 21, 32, 34, 51, 64]$
\end{enumerate}

\textbf{关键观察:}在第 $k$ 次迭代中,需要在 $n - k + 1$ 个元素的列表中找到最大值,这需要 $n - k$ 次比较。

\section{伪代码}

\begin{algorithm}
\caption{选择排序伪代码}
\begin{algorithmic}
\STATE \textbf{procedure} SelectionSort(A)
\STATE $n \leftarrow$ length of A
\FOR{$i = n$ \textbf{downto} $2$}
    \STATE $\text{max} \leftarrow i$  \COMMENT{假设第i个位置的元素最大}
    \FOR{$j = 1$ \textbf{to} $i - 1$}
        \IF{A[$j$] > A[max]}
            \STATE $\text{max} \leftarrow j$  \COMMENT{找到更大的元素}
        \ENDIF
    \ENDFOR
    \STATE swap(A[$i$], A[max])  \COMMENT{交换最大元素到第i个位置}
\ENDFOR
\STATE \textbf{end procedure}
\end{algorithmic}
\end{algorithm}

\section{时间复杂度分析}

\subsection{比较次数统计}

选择排序中的比较总数为:

\begin{equation}
\sum_{i=2}^{n} (i-1) = 1 + 2 + 3 + \cdots + (n-1) = \frac{n(n-1)}{2} = \Theta(n^2)
\end{equation}

\subsection{复杂度总结}

\begin{table}[h]
\centering
\begin{tabular}{|c|c|c|}
\hline
\textbf{复杂度指标} & \textbf{时间复杂度} & \textbf{说明} \\
\hline
最坏情况 & $\Theta(n^2)$ & 任何输入顺序都需要相同的比较次数 \\
\hline
最好情况 & $\Theta(n^2)$ & 不依赖于输入的初始顺序 \\
\hline
平均情况 & $\Theta(n^2)$ & 比较次数固定 \\
\hline
空间复杂度 & $O(1)$ & 只需要常数额外空间(原地排序) \\
\hline
交换次数 & $O(n)$ & 最多 $n-1$ 次交换 \\
\hline
\end{tabular}
\end{table}

\section{选择排序与插入排序的比较}

\subsection{选择排序的特性}

\begin{itemize}
    \item \textbf{位置确定性:}一旦找到一个元素的正确位置,该元素就不会再被移动
    \item \textbf{固定复杂度:}无论输入数据如何排序,都需要 $\Theta(n^2)$ 的比较次数
    \item \textbf{交换次数少:}最多进行 $n-1$ 次交换,比冒泡排序要少
    \item \textbf{不稳定:}相等元素的相对顺序可能改变
\end{itemize}

\subsection{插入排序的特性}

\begin{itemize}
    \item \textbf{自适应性:}从部分有序的输入中受益。若输入已排序,插入排序只需 $O(n)$ 时间
    \item \textbf{在线排序:}可以边接收数据边排序,无需等待所有数据都可用
    \item \textbf{稳定性:}相等元素的相对顺序保持不变
    \item \textbf{比较灵活:}在小数据集和几乎有序的数据上表现较好
\end{itemize}

\subsection{适用场景}

\begin{table}[h]
\centering
\begin{tabular}{|c|c|c|}
\hline
\textbf{场景} & \textbf{选择排序} & \textbf{插入排序} \\
\hline
完全无序的数据 & 较好 & 较好 \\
\hline
部分有序的数据 & 不好(仍需 $O(n^2)$) & 很好(可能 $O(n)$) \\
\hline
已排序的数据 & 不好(仍需 $O(n^2)$) & 很好(只需 $O(n)$) \\
\hline
实时处理数据 & 不好 & 很好 \\
\hline
稳定性要求 & 不适用 & 适用 \\
\hline
\end{tabular}
\end{table}

\section{算法优缺点总结}

\subsection{优点}

\begin{itemize}
    \item 算法简单,易于理解和实现
    \item 不需要额外的内存空间(原地排序)
    \item 交换次数较少,对内存写操作敏感的场景较好
    \item 比较次数固定,可预测性强
\end{itemize}

\subsection{缺点}

\begin{itemize}
    \item 时间复杂度为 $\Theta(n^2)$,对于大规模数据不适合
    \item 不是稳定排序
    \item 不能从部分有序的输入中获益
    \item 不能进行在线排序
\end{itemize}

\end{document}
