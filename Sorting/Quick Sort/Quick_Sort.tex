\documentclass[UTF8]{ctexart}
\usepackage[margin=1in]{geometry}
\usepackage{amsmath}
\usepackage{amssymb}
\usepackage{listings}
\usepackage{xcolor}
\usepackage{graphicx}
\usepackage{hyperref}

% 代码格式设置
\lstset{
    language=C++,
    basicstyle=\ttfamily\small,
    breaklines=true,
    numbers=left,
    numberstyle=\tiny,
    keywordstyle=\color{blue},
    commentstyle=\color{gray},
    stringstyle=\color{red},
    showstringspaces=false,
    backgroundcolor=\color{lightgray!20},
    frame=single
}

\title{\textbf{快速排序 (Quick Sort)}}
\author{}
\date{}

\begin{document}

\maketitle

\section{概述}
快速排序(Quick Sort)是一种高效的分治排序算法。它的基本思想是借鉴归并排序的分割思想,但避免了归并的代价。通过选择一个枢纽元素(pivot)将数组分成两个部分,使得左部分的所有元素都小于等于右部分的所有元素,然后递归地对两个部分进行排序。虽然最坏情况下时间复杂度为 $O(n^2)$,但在实践中,快速排序通常比其他 $O(n \log n)$ 的排序算法性能更好,是目前应用最广泛的排序算法之一。

\section{算法原理}

\subsection{基本思想}
快速排序的核心思想是:
\begin{enumerate}
    \item \textbf{分割(Partition)}:选择一个枢纽元素,将数组分成两部分,左部分的所有元素 $\leq$ 枢纽,右部分的所有元素 $\geq$ 枢纽
    \item \textbf{递归排序}:分别对左右两部分进行快速排序
    \item \textbf{合并}:由于分割后左右两部分已满足大小关系,分别排序后自动有序
\end{enumerate}

\subsection{与归并排序的比较}
与归并排序不同,快速排序:
\begin{itemize}
    \item 通过分割使得分割后的数据已经满足排序条件
    \item 无需合并步骤,因此省去了 $O(n)$ 的额外空间
    \item 是原地排序算法,空间复杂度为 $O(\log n)$(递归栈深度)
\end{itemize}

\section{分割操作详解}

\subsection{分割算法思想}
分割操作是快速排序的关键。基本策略如下:
\begin{enumerate}
    \item 选择数组的第一个元素作为枢纽 $v$
    \item 使用两个指针 $i$ 和 $j$,分别从左右两端向中间扫描
    \item 将 $\leq v$ 的元素移动到左侧,将 $> v$ 的元素移动到右侧
    \item 最终 $i$ 和 $j$ 相遇,分割完成
\end{enumerate}

\subsection{C++ 伪代码}
\begin{verbatim}
PARTITION (A, p, r) {
  v = A[p]                      // 第一个元素作为枢纽
  i = p - 1; 
  j = r + 1
  while (TRUE) {
    do  j--;  while (A[j] > v)  // j 从右向左扫描,找第一个 <= v
    do  i++;  while (A[i] < v)  // i 从左向右扫描,找第一个 > v
    if (i < j)                  // 未相遇,交换
      swap(A[i], A[j])
    else                        // 已相遇,分割完成
      return j
  }
}
\end{verbatim}

\subsection{分割示例}
原数组:$A = [5, 3, 2, 6, 4, 1, 3, 7]$,枢纽 $v = 5$

\begin{verbatim}
初始状态:
5 3 2 6 4 1 3 7
^ 枢纽

扫描过程(i 从左,j 从右):

第1步:j 扫描到 3 (<5),i 扫描到 6 (>5)
5 3 2 6 4 1 3 7
    i         j
交换后:
5 3 2 3 4 1 6 7

第2步:j 扫描到 1 (<5),i 扫描到 4 (>5)
5 3 2 3 4 1 6 7
      i   j
交换后:
5 3 2 1 4 3 6 7

第3步:j 扫描到 4 (>=5),i 扫描到 4
5 3 2 1 4 3 6 7
      i j
i < j 为假,分割完成

最终结果:
左部分:[3, 3, 2, 1, 4] <= 5
右部分:[6, 5, 7] > 5
\end{verbatim}

\section{C++ 实现}

\subsection{分割函数}
\begin{lstlisting}
#include <iostream>
#include <vector>
using namespace std;

// 分割函数:将数组分成两部分
// A[p..q]: 所有元素 <= v
// A[q+1..r]: 所有元素 > v
// 返回值:q 是分割点
int partition(vector<int>& arr, int p, int r) {
    int v = arr[p];  // 选择第一个元素作为枢纽
    int i = p - 1;
    int j = r + 1;
    
    while (true) {
        // j 从右向左扫描,找到第一个 <= v 的元素
        do {
            j--;
        } while (arr[j] > v);
        
        // i 从左向右扫描,找到第一个 > v 的元素
        do {
            i++;
        } while (arr[i] < v);
        
        // 如果 i < j,交换这两个元素
        if (i < j) {
            swap(arr[i], arr[j]);
        } else {
            // 否则分割完成,返回分割点
            return j;
        }
    }
}

// Lomuto 分割方案(另一种实现方式)
// 通常更容易理解,但性能不如 Hoare 分割
int partitionLomuto(vector<int>& arr, int low, int high) {
    int pivot = arr[high];
    int i = low - 1;
    
    for (int j = low; j < high; j++) {
        if (arr[j] < pivot) {
            i++;
            swap(arr[i], arr[j]);
        }
    }
    swap(arr[i + 1], arr[high]);
    return i + 1;
}
\end{lstlisting}

\subsection{快速排序主函数}
\begin{lstlisting}
// 快速排序的递归实现
// 时间复杂度(平均):O(n log n)
// 时间复杂度(最坏):O(n^2)
// 空间复杂度:O(log n)(递归栈)
void quickSort(vector<int>& arr, int p, int r) {
    if (p < r) {
        int q = partition(arr, p, r);
        quickSort(arr, p, q);           // 排序左部分
        quickSort(arr, q + 1, r);       // 排序右部分
    }
}

// 对外接口
void quickSortWrapper(vector<int>& arr) {
    if (!arr.empty()) {
        quickSort(arr, 0, arr.size() - 1);
    }
}

// 主程序示例
int main() {
    vector<int> arr = {5, 3, 2, 6, 4, 1, 3, 7};
    
    cout << "排序前:";
    for (int num : arr) {
        cout << num << " ";
    }
    cout << endl;
    
    quickSortWrapper(arr);
    
    cout << "排序后:";
    for (int num : arr) {
        cout << num << " ";
    }
    cout << endl;
    
    return 0;
}
\end{lstlisting}

\subsection{随机化快速排序}
\begin{lstlisting}
#include <random>

// 随机选择枢纽,避免最坏情况
int randomizedPartition(vector<int>& arr, int p, int r) {
    // 生成 [p, r] 之间的随机数
    random_device rd;
    mt19937 gen(rd());
    uniform_int_distribution<> dis(p, r);
    int randomIdx = dis(gen);
    
    // 将随机元素交换到第一个位置
    swap(arr[p], arr[randomIdx]);
    
    // 执行标准分割
    return partition(arr, p, r);
}

// 使用随机化分割的快速排序
void randomizedQuickSort(vector<int>& arr, int p, int r) {
    if (p < r) {
        int q = randomizedPartition(arr, p, r);
        randomizedQuickSort(arr, p, q);
        randomizedQuickSort(arr, q + 1, r);
    }
}
\end{lstlisting}

\section{复杂度分析}

\subsection{分割操作的复杂度}
分割操作的时间复杂度为 $O(n)$,其中 $n = r - p + 1$。
\begin{itemize}
    \item 指针 $i$ 和 $j$ 各遍历一次数组
    \item 最多进行 $\approx n/2$ 次交换
    \item 每次比较和交换都是常数时间
\end{itemize}

\subsection{最坏情况分析}

\subsubsection{最坏情况的发生}
最坏情况发生在每次分割都产生极不均衡的分割时,例如数组已经排序。

\textbf{示例:}排序已排序数组 $[1, 2, 3, 4, 5]$
\begin{itemize}
    \item 第1次分割:枢纽=1,左部分 0 个元素,右部分 4 个元素,代价 5 次比较
    \item 第2次分割:枢纽=2,左部分 0 个元素,右部分 3 个元素,代价 4 次比较
    \item 第3次分割:代价 3 次比较
    \item 第4次分割:代价 2 次比较
\end{itemize}

总成本:$5 + 4 + 3 + 2 = 14 = \frac{n(n+1)}{2} - 1$

\textbf{通用公式:}
$$T(n) = T(n-1) + n = T(n-2) + (n-1) + n = \cdots = \sum_{i=1}^{n} i = \frac{n(n+1)}{2} = \Theta(n^2)$$

\subsection{最好情况分析}

最好情况发生在每次分割都产生两个大小相等的部分时。

设 $T(n)$ 为排序 $n$ 个元素所需的比较次数:
$$T(n) = 2T(n/2) + n$$

递推求解:
\begin{align}
T(n) &= 2T(n/2) + n \\
&= 2[2T(n/4) + n/2] + n = 4T(n/4) + 2n \\
&= \cdots \\
&= 2^{\log_2 n} \cdot T(1) + n \log_2 n \\
&= n + n \log_2 n \\
&= \Theta(n \log n)
\end{align}

\subsection{平均情况分析}

\subsubsection{问题设定}
假设所有 $n$ 个元素互不相同且随机分布,枢纽可能是第 $1, 2, 3, \ldots, n$ 个元素,每个的概率为 $1/n$。

设 $T(n)$ 为快速排序 $n$ 个元素的平均比较次数。

\subsubsection{递推关系}
如果枢纽是第 $q$ 小的元素($1 \leq q \leq n$),则:
$$T(n) = T(q-1) + T(n-q) + n$$

平均情况下:
$$T(n) = \frac{1}{n} \sum_{q=1}^{n} [T(q-1) + T(n-q)] + n$$

\subsubsection{求解}
$$T(n) = \frac{1}{n} \sum_{q=1}^{n} [T(q-1) + T(n-q)] + n = \frac{2}{n} \sum_{i=0}^{n-1} T(i) + n$$

设 $nT(n) = \sum_{i=0}^{n} iT(i)$,则:
$$(n+1)T(n+1) = 2 \sum_{i=0}^{n} T(i) + (n+1)^2$$

化简可得:$T(n) = O(n \log n)$

\subsection{复杂度总结}
\begin{itemize}
    \item \textbf{最坏情况:} $O(n^2)$ \\
    发生于数组已排序或接近排序,枢纽选择不当时
    
    \item \textbf{平均情况:} $O(n \log n)$ \\
    大多数实际情况下的性能
    
    \item \textbf{最好情况:} $O(n \log n)$ \\
    每次分割均匀时
    
    \item \textbf{空间复杂度:} $O(\log n)$ \\
    递归调用栈的深度
\end{itemize}

\subsection{为什么快速排序在实践中最快?}
虽然快速排序和堆排序都是平均 $O(n \log n)$,但快速排序通常更快,因为:
\begin{itemize}
    \item 数据访问本地性好,缓存命中率高
    \item 常数因子小
    \item 使用了更少的比较和移动操作
    \item 对现代 CPU 的优化更友好
\end{itemize}

\section{枢纽元素的选择}

\subsection{问题与影响}
枢纽的选择对快速排序的性能有重大影响。选择不当可能导致最坏情况 $O(n^2)$。

\subsection{选择策略}

\subsubsection{策略1:选择第一个元素(最简单)}
\begin{itemize}
    \item 优点:实现简单
    \item 缺点:对于已排序或接近排序的数据,性能很差
\end{itemize}

\subsubsection{策略2:随机选择}
\begin{itemize}
    \item 优点:避免最坏情况,期望性能为 $O(n \log n)$
    \item 缺点:需要随机数生成器,略有性能开销
    \item 推荐:在处理不可信数据时使用
\end{itemize}

\subsubsection{策略3:三数取中(Median-of-three)}
从数组的三个位置(开始、中间、结束)选择元素,取其中位数:
\begin{lstlisting}
int medianOfThree(vector<int>& arr, int low, int mid, int high) {
    if (arr[low] > arr[mid]) swap(arr[low], arr[mid]);
    if (arr[mid] > arr[high]) swap(arr[mid], arr[high]);
    if (arr[low] > arr[mid]) swap(arr[low], arr[mid]);
    return mid;
}
\end{lstlisting}
\begin{itemize}
    \item 优点:避免在排序数据上的最坏情况
    \item 缺点:增加了常数开销
\end{itemize}

\section{快速选择算法 (QuickSelect)}

\subsection{问题}
在未排序的数组中找出第 $k$ 小的元素。

\subsection{思想}
借用快速排序的分割思想,但只递归处理包含第 $k$ 小元素的那一部分。

\subsection{C++ 实现}
\begin{lstlisting}
// 快速选择算法
// 返回数组中第 k 小的元素(0-indexed)
int quickSelect(vector<int>& arr, int low, int high, int k) {
    if (low == high) {
        return arr[low];
    }
    
    int q = partition(arr, low, high);
    
    // 如果 k 在左部分
    if (k <= q) {
        return quickSelect(arr, low, q, k);
    }
    // 如果 k 在右部分
    else {
        return quickSelect(arr, q + 1, high, k);
    }
}

// 使用示例
int main() {
    vector<int> arr = {5, 3, 2, 6, 4, 1, 3, 7};
    int k = 5;  // 找第 5 小的元素(0-indexed)
    
    int kthSmallest = quickSelect(arr, 0, arr.size() - 1, k);
    cout << "第 " << (k+1) << " 小的元素是:" << kthSmallest << endl;
    
    return 0;
}
\end{lstlisting}

\subsection{复杂度}
\begin{itemize}
    \item \textbf{平均情况:} $O(n)$ \\
    只需递归处理一部分数据
    
    \item \textbf{最坏情况:} $O(n^2)$ \\
    枢纽选择不当时
\end{itemize}

\section{排序的下界}

\subsection{问题}
任何基于比较的排序算法至少需要多少次比较?

\subsection{决策树模型}
可以用二叉决策树表示一个排序算法:
\begin{itemize}
    \item \textbf{节点}:表示比较 $A[i]$ 和 $A[j]$
    \item \textbf{边}:表示比较的结果($<$、$=$、$>$)
    \item \textbf{叶子}:表示最终的排序结果(某个排列)
\end{itemize}

\subsection{下界证明}

\subsubsection{关键观察}
对 $n$ 个元素的排序:
\begin{itemize}
    \item 共有 $n!$ 种可能的初始排列
    \item 每个排列对应决策树的一个叶子
    \item 因此决策树至少有 $n!$ 个叶子
\end{itemize}

\subsubsection{二叉树的性质}
\textbf{引理:}高度为 $h$ 的二叉树最多有 $2^h$ 个叶子。

\textbf{证明:}由归纳法,$h=0$ 时有 1 个叶子;$h=k$ 时递推。

\subsubsection{推导下界}
如果一个排序算法的最坏情况比较次数为 $T(n)$,则对应决策树的高度为 $T(n)$。

由于至少需要 $n!$ 个叶子:
$$2^{T(n)} \geq n!$$

两边取对数:
$$T(n) \geq \log_2(n!)$$

\subsubsection{计算 $\log_2(n!)$ 的下界}
$$n! = 1 \cdot 2 \cdot 3 \cdots n$$

对于 $1 \leq k \leq n$,有 $k \leq n$,因此:
$$n! = \prod_{k=1}^{n} k \leq n^n$$

$$n! = \prod_{k=1}^{n} k \geq \prod_{k=n/2}^{n} k \geq (n/2)^{n/2}$$

取对数:
$$\log_2(n!) \geq \frac{n}{2} \log_2(n/2) = \frac{n}{2}(\log_2 n - 1) = \frac{n \log_2 n}{2} - \frac{n}{2}$$

更精确地(使用 Stirling 公式):
$$\log_2(n!) = \Theta(n \log_2 n)$$

\subsection{排序下界定理}
\textbf{定理:}任何基于比较的排序算法在最坏情况下至少需要 $\Theta(n \log_2 n)$ 次比较。

\textbf{推论:}
\begin{itemize}
    \item 归并排序、堆排序达到了这个下界,是最优的
    \item 快速排序的平均情况也达到了这个下界
    \item 不存在更好的基于比较的排序算法
\end{itemize}

\subsection{平均情况下界}
决策树中任意叶子的平均深度也是 $\Theta(n \log_2 n)$,因此任何比较型排序算法的平均情况复杂度下界为 $\Theta(n \log_2 n)$。

\section{特点与应用}

\subsection{优点}
\begin{itemize}
    \item 平均时间复杂度为 $O(n \log n)$,且常数因子小
    \item 原地排序,空间复杂度只有 $O(\log n)$
    \item 在实践中性能优于大多数其他排序算法
    \item 可以通过不同的枢纽选择策略进行优化
    \item 适合大规模数据的排序
\end{itemize}

\subsection{缺点}
\begin{itemize}
    \item 最坏情况时间复杂度为 $O(n^2)$
    \item 不是稳定的排序算法(相等元素的相对顺序可能改变)
    \item 对于已排序或接近排序的数据,性能较差(除非使用好的枢纽选择策略)
    \item 枢纽选择不当容易陷入最坏情况
\end{itemize}

\subsection{应用场景}
\begin{itemize}
    \item 大多数编程语言和库的默认排序算法(例如 C++ 的 std::sort)
    \item 需要原地排序的场景(空间受限)
    \item 大规模数据排序
    \item 需要快速响应的实时系统
\end{itemize}

\end{document}
