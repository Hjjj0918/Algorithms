\documentclass[UTF8]{ctexart}
\usepackage[margin=1in]{geometry}
\usepackage{amsmath}
\usepackage{amssymb}
\usepackage{listings}
\usepackage{xcolor}
\usepackage{graphicx}
\usepackage{hyperref}
\usepackage{algorithmic}
\usepackage{algorithm}

\title{二叉搜索树 (Binary Search Tree)}
\author{}
\date{}

\begin{document}

\maketitle

\section{二叉树基础}

\subsection{二叉树定义}
二叉树是一种树形数据结构,其中每个节点最多有两个子节点,通常称为左子节点和右子节点。

二叉树的重要性质:
\begin{itemize}
    \item 一棵深度为 $k$ 的二叉树最多有 $2^k - 1$ 个节点
    \item 一棵有 $n$ 个节点的二叉树的高度至少为 $\log_2(n+1)$
    \item 满二叉树:深度为 $k$ 且有 $2^k - 1$ 个节点
    \item 完全二叉树:除最后一层外,其他层均满,最后一层从左到右连续
\end{itemize}

\subsection{二叉树的三种遍历方式}

\subsubsection{1. 前序遍历 (Pre-order Traversal)}
访问顺序:根节点 → 左子树 → 右子树

\begin{algorithm}
\caption{前序遍历伪代码}
\begin{algorithmic}
\STATE \textbf{procedure} PreOrder(node)
\IF{node is not null}
    \STATE visit(node)  \COMMENT{访问当前节点}
    \STATE PreOrder(node.left)  \COMMENT{遍历左子树}
    \STATE PreOrder(node.right)  \COMMENT{遍历右子树}
\ENDIF
\STATE \textbf{end procedure}
\end{algorithmic}
\end{algorithm}

\subsubsection{2. 中序遍历 (In-order Traversal)}
访问顺序:左子树 → 根节点 → 右子树

\begin{algorithm}
\caption{中序遍历伪代码}
\begin{algorithmic}
\STATE \textbf{procedure} InOrder(node)
\IF{node is not null}
    \STATE InOrder(node.left)  \COMMENT{遍历左子树}
    \STATE visit(node)  \COMMENT{访问当前节点}
    \STATE InOrder(node.right)  \COMMENT{遍历右子树}
\ENDIF
\STATE \textbf{end procedure}
\end{algorithmic}
\end{algorithm}

\subsubsection{3. 后序遍历 (Post-order Traversal)}
访问顺序:左子树 → 右子树 → 根节点

\begin{algorithm}
\caption{后序遍历伪代码}
\begin{algorithmic}
\STATE \textbf{procedure} PostOrder(node)
\IF{node is not null}
    \STATE PostOrder(node.left)  \COMMENT{遍历左子树}
    \STATE PostOrder(node.right)  \COMMENT{遍历右子树}
    \STATE visit(node)  \COMMENT{访问当前节点}
\ENDIF
\STATE \textbf{end procedure}
\end{algorithmic}
\end{algorithm}

\section{二叉搜索树 (BST)}

\subsection{定义}
二叉搜索树是一种特殊的二叉树,满足以下性质:
\begin{itemize}
    \item 对于任意节点,其左子树中所有节点的值都小于该节点的值
    \item 对于任意节点,其右子树中所有节点的值都大于该节点的值
    \item 左右子树也都是二叉搜索树
\end{itemize}

\subsection{查找操作}

\begin{algorithm}
\caption{BST查找操作伪代码}
\begin{algorithmic}
\STATE \textbf{procedure} Search(root, key)
\STATE node $\leftarrow$ root
\WHILE{node is not null}
    \IF{node.value == key}
        \STATE \textbf{return} node  \COMMENT{找到目标}
    \ELSIF{key < node.value}
        \STATE node $\leftarrow$ node.left  \COMMENT{在左子树中查找}
    \ELSE
        \STATE node $\leftarrow$ node.right  \COMMENT{在右子树中查找}
    \ENDIF
\ENDWHILE
\STATE \textbf{return} null  \COMMENT{未找到}
\STATE \textbf{end procedure}
\end{algorithmic}
\end{algorithm}

时间复杂度:平均情况 $O(\log n)$,最坏情况 $O(n)$

\subsection{插入操作}

\begin{algorithm}
\caption{BST插入操作伪代码}
\begin{algorithmic}
\STATE \textbf{procedure} Insert(root, value)
\IF{root is null}
    \STATE root $\leftarrow$ new Node(value)
    \STATE \textbf{return} root
\ENDIF
\IF{value < root.value}
    \STATE root.left $\leftarrow$ Insert(root.left, value)
\ELSIF{value > root.value}
    \STATE root.right $\leftarrow$ Insert(root.right, value)
\ELSE
    \STATE \textbf{return} root  \COMMENT{重复值不插入}
\ENDIF
\STATE \textbf{return} root
\STATE \textbf{end procedure}
\end{algorithmic}
\end{algorithm}

\subsection{删除操作}

删除操作分为三种情况:

\begin{algorithm}
\caption{BST删除操作伪代码}
\begin{algorithmic}
\STATE \textbf{procedure} Delete(root, value)
\IF{root is null}
    \STATE \textbf{return} null
\ENDIF
\IF{value < root.value}
    \STATE root.left $\leftarrow$ Delete(root.left, value)
\ELSIF{value > root.value}
    \STATE root.right $\leftarrow$ Delete(root.right, value)
\ELSE
    \COMMENT{找到要删除的节点}
    \IF{root.left is null}
        \STATE \textbf{return} root.right  \COMMENT{情况1:无左子树}
    \ELSIF{root.right is null}
        \STATE \textbf{return} root.left  \COMMENT{情况2:无右子树}
    \ELSE
        \COMMENT{情况3:有两个子树}
        \STATE minNode $\leftarrow$ FindMin(root.right)  \COMMENT{找右子树最小值}
        \STATE root.value $\leftarrow$ minNode.value  \COMMENT{用最小值替换}
        \STATE root.right $\leftarrow$ Delete(root.right, minNode.value)
    \ENDIF
\ENDIF
\STATE \textbf{return} root
\STATE \textbf{end procedure}
\end{algorithmic}
\end{algorithm}

\subsubsection{辅助函数:寻找最小值}

\begin{algorithm}
\caption{寻找最小值节点伪代码}
\begin{algorithmic}
\STATE \textbf{procedure} FindMin(root)
\IF{root is null}
    \STATE \textbf{return} null
\ENDIF
\WHILE{root.left is not null}
    \STATE root $\leftarrow$ root.left
\ENDWHILE
\STATE \textbf{return} root
\STATE \textbf{end procedure}
\end{algorithmic}
\end{algorithm}

\section{BST应用}

\subsection{1. 判断一个二叉树是否为二叉搜索树}

该问题的关键是确保每个节点的值在有效范围内。

\subsubsection{1.1 方法一:范围检查}

\begin{algorithm}
\caption{验证BST伪代码(范围检查)}
\begin{algorithmic}
\STATE \textbf{procedure} IsValidBST(root, minVal, maxVal)
\IF{root is null}
    \STATE \textbf{return} true
\ENDIF
\IF{root.value $\leq$ minVal OR root.value $\geq$ maxVal}
    \STATE \textbf{return} false  \COMMENT{违反BST性质}
\ENDIF
\STATE leftValid $\leftarrow$ IsValidBST(root.left, minVal, root.value)
\STATE rightValid $\leftarrow$ IsValidBST(root.right, root.value, maxVal)
\STATE \textbf{return} leftValid AND rightValid
\STATE \textbf{end procedure}
\end{algorithmic}
\end{algorithm}

初始调用:\texttt{IsValidBST(root, $-\infty$, $+\infty$)}

时间复杂度:$O(n)$,其中 $n$ 为节点数

\subsubsection{1.2 方法二:中序遍历验证}

在BST上进行中序遍历应该得到递增序列。

\begin{algorithm}
\caption{验证BST伪代码(中序遍历)}
\begin{algorithmic}
\STATE \textbf{procedure} IsValidBST(root)
\STATE prevVal $\leftarrow$ $-\infty$
\STATE valid $\leftarrow$ InOrderValidate(root, prevVal)
\STATE \textbf{return} valid
\STATE \textbf{end procedure}

\STATE \textbf{procedure} InOrderValidate(node, prevVal)
\IF{node is null}
    \STATE \textbf{return} true
\ENDIF
\STATE leftValid $\leftarrow$ InOrderValidate(node.left, prevVal)
\IF{NOT leftValid}
    \STATE \textbf{return} false
\ENDIF
\IF{node.value $\leq$ prevVal}
    \STATE \textbf{return} false  \COMMENT{不满足递增条件}
\ENDIF
\STATE prevVal $\leftarrow$ node.value
\STATE rightValid $\leftarrow$ InOrderValidate(node.right, prevVal)
\STATE \textbf{return} rightValid
\STATE \textbf{end procedure}
\end{algorithmic}
\end{algorithm}

时间复杂度:$O(n)$,其中 $n$ 为节点数

\subsection{2. BST的最近公共祖先}

\subsubsection{2.1 伪代码实现}

\begin{algorithm}
\caption{找两个节点的最近公共祖先伪代码}
\begin{algorithmic}
\STATE \textbf{procedure} FindLCA(root, p, q)
\IF{root is null}
    \STATE \textbf{return} null
\ENDIF
\IF{p.value < root.value AND q.value < root.value}
    \STATE \textbf{return} FindLCA(root.left, p, q)  \COMMENT{都在左子树}
\ELSIF{p.value > root.value AND q.value > root.value}
    \STATE \textbf{return} FindLCA(root.right, p, q)  \COMMENT{都在右子树}
\ELSE
    \STATE \textbf{return} root  \COMMENT{分别在两侧或为根节点}
\ENDIF
\STATE \textbf{end procedure}
\end{algorithmic}
\end{algorithm}

时间复杂度:$O(\log n)$ 平均情况,$O(n)$ 最坏情况

\subsection{3. 找BST中第K小的元素}

\subsubsection{3.1 伪代码实现}

\begin{algorithm}
\caption{找第K小元素伪代码}
\begin{algorithmic}
\STATE \textbf{procedure} KthSmallest(root, k)
\STATE stack $\leftarrow$ empty stack
\STATE count $\leftarrow$ 0
\STATE node $\leftarrow$ root
\WHILE{stack is not empty OR node is not null}
    \WHILE{node is not null}
        \STATE stack.push(node)
        \STATE node $\leftarrow$ node.left
    \ENDWHILE
    \STATE node $\leftarrow$ stack.pop()
    \STATE count $\leftarrow$ count + 1
    \IF{count == k}
        \STATE \textbf{return} node.value
    \ENDIF
    \STATE node $\leftarrow$ node.right
\ENDWHILE
\STATE \textbf{return} null
\STATE \textbf{end procedure}
\end{algorithmic}
\end{algorithm}

时间复杂度:$O(k)$(最坏情况 $O(n)$)

\section{BST 例题及伪代码}

\subsection{例题1:求BST所有元素的和}

\textbf{问题:}给定一个二叉搜索树,求树中所有节点值的和。

\textbf{思路:}递归遍历整棵树,将每个节点的值加起来。

\begin{algorithm}
\caption{求BST所有元素的和}
\begin{algorithmic}
\STATE \textbf{procedure} SumOfTree(root)
\IF{root is null}
    \STATE \textbf{return} 0
\ENDIF
\STATE leftSum $\leftarrow$ SumOfTree(root.left)  \COMMENT{左子树的和}
\STATE rightSum $\leftarrow$ SumOfTree(root.right)  \COMMENT{右子树的和}
\STATE \textbf{return} root.value + leftSum + rightSum  \COMMENT{当前节点加左右子树}
\STATE \textbf{end procedure}
\end{algorithmic}
\end{algorithm}

\textbf{示例:}对于树
\begin{verbatim}
        5
       / \
      3   7
     / \   \
    2   4   8
\end{verbatim}
SumOfTree 返回 $5+3+7+2+4+8=29$。

时间复杂度:$O(n)$(访问每个节点一次);空间复杂度:$O(h)$(递归栈深度)

\subsection{例题2:求范围[a, b]内元素的和}

\textbf{问题:}给定一个二叉搜索树和两个整数 $a, b$($a \leq b$),求树中所有值在 $[a, b]$ 范围内的节点值之和。

\textbf{思路:}利用BST的性质,只遍历值在范围内或可能在范围内的子树,避免遍历无关分支。

\begin{algorithm}
\caption{求范围[a,b]内元素的和}
\begin{algorithmic}
\STATE \textbf{procedure} RangeSumBST(root, a, b)
\IF{root is null}
    \STATE \textbf{return} 0
\ENDIF
\IF{root.value < a}
    \COMMENT{当前节点值小于a,左子树更小,只需查看右子树}
    \STATE \textbf{return} RangeSumBST(root.right, a, b)
\ELSIF{root.value > b}
    \COMMENT{当前节点值大于b,右子树更大,只需查看左子树}
    \STATE \textbf{return} RangeSumBST(root.left, a, b)
\ELSE
    \COMMENT{当前节点在范围[a,b]内}
    \STATE leftSum $\leftarrow$ RangeSumBST(root.left, a, b)
    \STATE rightSum $\leftarrow$ RangeSumBST(root.right, a, b)
    \STATE \textbf{return} root.value + leftSum + rightSum
\ENDIF
\STATE \textbf{end procedure}
\end{algorithmic}
\end{algorithm}

\textbf{示例:}对于上面的树,求范围 $[3,7]$ 内的和:
\begin{itemize}
    \item 5 在范围内,加入结果
    \item 对 3 查询 $[3,7]$:3 在范围内,加入结果;4 在范围内,加入结果;2 不在范围内,跳过
    \item 对 7 查询 $[3,7]$:7 在范围内,加入结果;8 不在范围内,跳过
    \item 最终结果:$5+3+4+7=19$
\end{itemize}

时间复杂度:$O(n)$ 最坏情况(当范围覆盖全树);平均情况下更优;空间复杂度:$O(h)$

\subsection{例题3:求BST的最小差值}

\textbf{问题:}给定一个二叉搜索树,求树中任意两个不同节点值的差的最小值。

\textbf{思路:}在BST上做中序遍历得到递增序列,最小差值必定出现在相邻元素之间。

\begin{algorithm}
\caption{求BST的最小差值}
\begin{algorithmic}
\STATE \textbf{procedure} MinDifferenceBST(root)
\STATE prev $\leftarrow$ null
\STATE minDiff $\leftarrow$ $\infty$
\STATE InOrderTraversal(root, prev, minDiff)
\STATE \textbf{return} minDiff
\STATE \textbf{end procedure}

\STATE \textbf{procedure} InOrderTraversal(node, prev, minDiff)
\IF{node is null}
    \STATE \textbf{return}
\ENDIF
\STATE InOrderTraversal(node.left, prev, minDiff)  \COMMENT{遍历左子树}
\IF{prev is not null}
    \STATE minDiff $\leftarrow$ min(minDiff, node.value - prev)  \COMMENT{更新最小差值}
\ENDIF
\STATE prev $\leftarrow$ node.value  \COMMENT{更新前驱节点}
\STATE InOrderTraversal(node.right, prev, minDiff)  \COMMENT{遍历右子树}
\STATE \textbf{end procedure}
\end{algorithmic}
\end{algorithm}

\textbf{示例:}对于树 2-1-3,中序遍历为 1, 2, 3,差值为 $\{2-1=1, 3-2=1\}$,最小差值为 1。

时间复杂度:$O(n)$;空间复杂度:$O(h)$

\section{BST的复杂度分析}

\begin{table}[h]
\centering
\begin{tabular}{|c|c|c|}
\hline
\textbf{操作} & \textbf{平均时间复杂度} & \textbf{最坏时间复杂度} \\
\hline
查找 & $O(\log n)$ & $O(n)$ \\
\hline
插入 & $O(\log n)$ & $O(n)$ \\
\hline
删除 & $O(\log n)$ & $O(n)$ \\
\hline
遍历 & $O(n)$ & $O(n)$ \\
\hline
\end{tabular}
\end{table}

\textbf{注:}最坏情况发生在树退化为链表时(例如,顺序插入有序数据)。

\end{document}
