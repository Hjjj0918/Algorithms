\documentclass[UTF8]{ctexart}
\usepackage[margin=1in]{geometry}
\usepackage{amsmath}
\usepackage{amssymb}
\usepackage{listings}
\usepackage{xcolor}
\usepackage{graphicx}
\usepackage{hyperref}

% 代码格式设置
\lstset{
    language=C++,
    basicstyle=\ttfamily\small,
    breaklines=true,
    numbers=left,
    numberstyle=\tiny,
    keywordstyle=\color{blue},
    commentstyle=\color{gray},
    stringstyle=\color{red},
    showstringspaces=false,
    backgroundcolor=\color{lightgray!20},
    frame=single
}

\title{\textbf{跳跃搜索与平方根搜索 (Jump Search \& Square-Root Search)}}
\author{}
\date{}

\begin{document}

\maketitle

\section{搜索问题的基本模型}

\subsection{问题定义}
给定包含 $n$ 个元素的列表 $L$,和目标元素 $X$(由键 $k$ 标识),我们需要:
\begin{itemize}
  \item 判断 $X$ 是否在 $L$ 中
  \item 如果存在,检索该元素
\end{itemize}

我们假设搜索算法的运行时间主要由\textbf{元素之间的比较操作}次数决定。

\subsection{任何搜索算法的下界}

\paragraph{问题:最坏情况下,任何搜索算法都需要至少 $n$ 次比较吗?}

\paragraph{证明(反证法)}
假设某搜索算法只进行 $n-1$ 次(或更少)比较。由于列表 $L$ 中有 $n$ 个元素,算法必然避免了与至少一个元素 $L[i]$ 的比较。

构造特殊的输入实例:\textbf{故意将 $X$ 放在位置 $L[i]$}。

当算法处理这个新的列表时,由于从未与 $L[i]$ 进行比较,算法会错误地报告"$X$ 不在 $L$ 中"。因此算法不正确。

\textbf{结论:}没有任何搜索算法的最坏情况比较次数少于 $n$。

\paragraph{关键假设}
上述论证基于两个假设:
\begin{enumerate}
  \item 算法对所有可能的输入都跳过 $L[i]$
  \item 如果算法跳过 $L[i]$,则无法从 $L[i]$ 中获得任何信息
\end{enumerate}

\paragraph{突破下界的条件}
如果数据\textbf{具有特定的结构}(如有序),上述假设就不成立。在这种情况下,我们可能突破 $O(n)$ 的下界。

\section{跳跃搜索(Jump Search)}

\subsection{基本思想}
线性搜索逐个检查列表中的元素。跳跃搜索采用不同策略:

假设 $L$ 按键值升序排列。我们\textbf{跳过}某些元素进行检查,而不是逐个扫描。

\subsubsection{简单例子}
比较 $X$ 与 $L[5]$:
\begin{itemize}
  \item 若 $L[5] = X$,完成
  \item 若 $L[5] > X$,我们只需考虑 $L[1], L[2], L[3], L[4]$(前 4 个元素)
  \item 若 $L[5] < X$,我们知道 $X > L[1], L[2], L[3], L[4]$(跳过了 4 个元素的逐个比较)
\end{itemize}

这种策略通过\textbf{剪枝(pruning)}——组织数据或计算,使我们能跳过不必要的元素——比线性搜索更高效。

\subsection{跳跃搜索的算法}

\subsubsection{参数与策略}
选择跳跃步长 $k < n$,进行以下步骤:

\begin{enumerate}
  \item 比较 $X$ 与位置 $k, 2k, 3k, \ldots$ 的元素,直到找到某个 $i$ 使得:
  \[
  X < L[ik] \text{ 或 } X = L[ik]
  \]
  
  \item 若 $X = L[ik]$,搜索成功
  
  \item 否则,在子列表 $L[(i-1)k+1], \ldots, L[ik-1]$ 上进行\textbf{线性搜索}
  
  \item 若 $X$ 大于最后一个跳跃位置,则在最后的子列表 $L[\lceil n/k \rceil k + 1, \ldots, n]$ 上进行线性搜索
\end{enumerate}

\subsection{复杂度分析}

\paragraph{最坏情况比较次数}
设 $m = \lceil n/k \rceil$ 为跳跃停止点的数量。假设 $mk \leq n < (m+1)k$,则:
\begin{itemize}
  \item 定位目标段需要最多 $m$ 次比较
  \item 在段内进行线性搜索需要最多 $k-1$ 次比较
\end{itemize}

因此,最坏情况下的比较次数为:
\[
f(n, k) = \lceil n/k \rceil + k - 1
\]

\paragraph{数值示例}
对 $n = 1000$,不同 $k$ 值的比较次数:

\begin{center}
\begin{tabular}{|c|c|c|c|c|c|}
\hline
$k$ & 3 & 10 & 30 & 50 & 100 \\
\hline
$f(1000, k)$ & 335 & 109 & 62 & 69 & 110 \\
\hline
\end{tabular}
\end{center}

观察:最优的 $k$ 值约为 $30 \sim 50$。

\section{平方根搜索(Square-Root Search)}

\subsection{最优跳跃步长}

\paragraph{优化问题}
我们要最小化 $f(n, k) = \lceil n/k \rceil + k - 1$。

对 $f(n, k)$ 关于 $k$ 求导:
\[
\frac{d}{dk}\left(\frac{n}{k} + k - 1\right) = -\frac{n}{k^2} + 1
\]

令导数为零:
\[
1 - \frac{n}{k^2} = 0 \quad \Rightarrow \quad k = \sqrt{n}
\]

\paragraph{最优步长}
最优的跳跃步长约为 $k \approx \sqrt{n}$。

对于 $n = 1000$:
\[
k \approx \sqrt{1000} \approx 31.62 \approx 32
\]

此时:
\[
f(1000, 32) = \lceil 1000/32 \rceil + 32 - 1 = 32 + 31 = 63
\]

相比线性搜索的最坏情况(1000 次比较),改进显著。

\subsection{平方根搜索的定义与复杂度}

\paragraph{定义}
采用步长 $k = \sqrt{n}$ 的跳跃搜索称为\textbf{平方根搜索}。

\paragraph{最坏情况复杂度}
\[
f\left(n, \sqrt{n}\right) = \frac{n}{\sqrt{n}} + \sqrt{n} - 1 = \sqrt{n} + \sqrt{n} - 1 = 2\sqrt{n} - 1
\]

因此,平方根搜索的最坏情况时间复杂度为:
\[
\boxed{O(\sqrt{n})}
\]

\subsection{性能对比}

\begin{center}
\begin{tabular}{|c|c|c|c|}
\hline
搜索算法 & 前提条件 & 最坏情况复杂度 & 比较次数($n=1000$) \\
\hline
线性搜索 & 无 & $O(n)$ & 1000 \\
\hline
二分查找 & 有序 & $O(\log n)$ & 10 \\
\hline
跳跃搜索(最优 $k$) & 有序 & $O(\sqrt{n})$ & 63 \\
\hline
平方根搜索 & 有序 & $O(\sqrt{n})$ & 63 \\
\hline
\end{tabular}
\end{center}

\section{C++ 实现}

\subsection{跳跃搜索}
\begin{lstlisting}
#include <iostream>
#include <cmath>
using namespace std;

int jumpSearch(const int* array, int size, int key) {
    int k = (int)sqrt(size);  // 最优步长为 sqrt(n)
    int i = k;
    int prev = 0;
    
    // 跳跃阶段:找到目标所在的段
    while (i < size && array[i] < key) {
        prev = i;
        i += k;
    }
    
    // 线性搜索阶段:在段内查找
    while (array[prev] < key && prev < i && prev < size) {
        prev++;
    }
    
    // 检查是否找到
    if (prev < size && array[prev] == key) {
        return prev;
    }
    
    return -1;  // 未找到
}

int main() {
    int arr[] = {0, 1, 2, 3, 4, 5, 6, 7, 8, 9,
                 10, 11, 12, 13, 14, 15, 20, 25};
    int size = sizeof(arr) / sizeof(arr[0]);
    int key = 13;
    
    int result = jumpSearch(arr, size, key);
    
    if (result != -1) {
        cout << "Element " << key << " found at index "
             << result << endl;
    } else {
        cout << "Element " << key << " not found" << endl;
    }
    
    return 0;
}
\end{lstlisting}

\section{总结}

\subsection{关键洞察}
\begin{itemize}
  \item \textbf{无序数据}:任何搜索算法最坏情况下需要 $n$ 次比较(信息论下界)
  \item \textbf{有序数据}:可以利用结构信息进行剪枝,突破 $O(n)$ 下界
  \item \textbf{跳跃搜索}:结合跳跃与线性搜索,复杂度为 $O(\sqrt{n})$(当 $k=\sqrt{n}$)
  \item \textbf{二分搜索}:在有序数据上性能更优,复杂度为 $O(\log n)$
\end{itemize}

\subsection{何时使用}
\begin{itemize}
  \item \textbf{平方根搜索}:当二分查找不适用(如链表),但数据有序时
  \item \textbf{跳跃搜索}:教学目的,演示如何通过剪枝改进搜索
  \item \textbf{二分查找}:大多数有序数组的实际应用
\end{itemize}

\end{document}
