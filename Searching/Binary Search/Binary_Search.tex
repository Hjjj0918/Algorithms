\documentclass[UTF8]{ctexart}
\usepackage[margin=1in]{geometry}
\usepackage{amsmath}
\usepackage{amssymb}
\usepackage{listings}
\usepackage{xcolor}
\usepackage{graphicx}
\usepackage{hyperref}

% 代码格式设置
\lstset{
    language=C++,
    basicstyle=\ttfamily\small,
    breaklines=true,
    numbers=left,
    numberstyle=\tiny,
    keywordstyle=\color{blue},
    commentstyle=\color{gray},
    stringstyle=\color{red},
    showstringspaces=false,
    backgroundcolor=\color{lightgray!20},
    frame=single
}

\title{\textbf{二分查找 (Binary Search)}}
\author{}
\date{}

\begin{document}

\maketitle

\section{引言}

\subsection{问题回顾}
在跳跃搜索中,我们提出了一个问题:为了在最坏情况下消除最大的列表段(即无论多么不幸都能消除大段),我们应该选择什么样的策略?

\subsection{答案}
\textbf{选择中间元素进行比较。}

无论比较结果如何,我们都能消除列表的\textbf{一半}。通过递归地应用这个策略到剩余的段,直到找到目标元素或段缩小为一个元素为止——这就是\textbf{二分查找}。

\section{算法思想}

二分查找(Binary Search)是一种高效的搜索算法,适用于\textbf{有序数组}。其核心思想是:
\begin{enumerate}
  \item 比较目标元素 $X$ 与中间位置的元素
  \item 根据比较结果,决定在左半部分还是右半部分继续搜索
  \item 重复此过程,每次消除一半的搜索空间
\end{enumerate}

\section{伪代码}

\begin{lstlisting}
BinarySearch(L, low, high, X)
/* 考虑的段为 L[low, high] */
while (high >= low) {
    mid = floor((low + high) / 2)
    case L[mid] {
        < X: low = mid + 1;      /* 目标在右半部分 */
        = X: return(mid);         /* 找到目标 */
        > X: high = mid - 1;      /* 目标在左半部分 */
    }
return ("not found!")             /* 搜索失败 */
\end{lstlisting}

\section{最坏情况复杂度分析}

\subsection{递推关系}
设 $f(n)$ 为二分查找在最坏情况下检查的元素个数。建立如下递推关系:

\[
f(n) = \begin{cases}
1, & \text{if } n = 1 \\
f(\lfloor n/2 \rfloor) + 1, & \text{if } n > 1
\end{cases}
\]

\subsection{展开递推}
反复展开递推关系:
\begin{align}
f(n) &= f(\lfloor n/2 \rfloor) + 1 \\
&= f(\lfloor \lfloor n/2 \rfloor / 2 \rfloor) + 2 \\
&= f(\lfloor n/4 \rfloor) + 2 \\
&= f(\lfloor n/8 \rfloor) + 3 \\
&\vdots \\
&= f(\lfloor n/2^i \rfloor) + i
\end{align}

\paragraph{注}
可以验证 $\lfloor \lfloor n/2 \rfloor / 2 \rfloor = \lfloor n/4 \rfloor$(有兴趣的读者自证)。

\subsection{求解}
当 $\lfloor n/2^i \rfloor = 1$ 时,递推停止。此时:
\[
2^i = n \quad \Rightarrow \quad i = \log_2 n
\]

因此:
\[
f(n) = f(1) + \log_2 n = 1 + \log_2 n
\]

即 $f(n) = \lfloor \log_2 n \rfloor + 1$。

\paragraph{最坏情况复杂度}
\[
\boxed{f(n) = O(\log n)}
\]

\section{平均情况复杂度分析}

\subsection{不成功的搜索}
当列表中不存在目标元素时,二分查找必须将搜索段缩小到无法继续,此时比较次数与最坏情况相同,为 $O(\log n)$。

\subsection{成功的搜索}
当列表中存在目标元素时,平均查找次数为 $O(\log n)$。

这是因为:
\begin{itemize}
  \item 约有 $1/2$ 的概率在第一次比较时找到(在中间位置)
  \item 约有 $1/4$ 的概率在第二次比较时找到(在子列表中间位置)
  \item 约有 $1/8$ 的概率在第三次比较时找到
  \item \ldots
\end{itemize}

期望比较次数为:
\[
\mathbb{E}[f] = 1 \cdot \frac{1}{2} + 2 \cdot \frac{1}{4} + 3 \cdot \frac{1}{8} + \cdots = \sum_{i=1}^{\log_2 n} i \cdot \frac{1}{2^i} = O(\log n)
\]

\section{C++ 实现}

\subsection{迭代版本}
\begin{lstlisting}
#include <iostream>
using namespace std;

int binarySearchIterative(const int* array, int size, int key) {
    int low = 0;
    int high = size - 1;
    
    while (low <= high) {
        int mid = (low + high) / 2;
        
        if (array[mid] == key) {
            return mid;          /* 找到目标 */
        } else if (array[mid] < key) {
            low = mid + 1;       /* 在右半部分查找 */
        } else {
            high = mid - 1;      /* 在左半部分查找 */
        }
    }
    
    return -1;                   /* 未找到 */
}
\end{lstlisting}

\subsection{递归版本}
\begin{lstlisting}
int binarySearchRecursive(const int* array, int low, int high, int key) {
    if (low > high) {
        return -1;               /* 未找到 */
    }
    
    int mid = (low + high) / 2;
    
    if (array[mid] == key) {
        return mid;              /* 找到目标 */
    } else if (array[mid] < key) {
        return binarySearchRecursive(array, mid + 1, high, key);
    } else {
        return binarySearchRecursive(array, low, mid - 1, key);
    }
}
\end{lstlisting}

\subsection{完整示例程序}
\begin{lstlisting}
#include <iostream>
using namespace std;

int binarySearch(const int* array, int size, int key) {
    int low = 0;
    int high = size - 1;
    
    while (low <= high) {
        int mid = (low + high) / 2;
        
        if (array[mid] == key) {
            return mid;
        } else if (array[mid] < key) {
            low = mid + 1;
        } else {
            high = mid - 1;
        }
    }
    
    return -1;
}

int main() {
    int arr[] = {1, 3, 5, 7, 9, 11, 13, 15, 17, 19};
    int size = sizeof(arr) / sizeof(arr[0]);
    int key = 13;
    
    int result = binarySearch(arr, size, key);
    
    if (result != -1) {
        cout << "Element " << key << " found at index "
             << result << endl;
    } else {
        cout << "Element " << key << " not found" << endl;
    }
    
    return 0;
}
\end{lstlisting}

\section{与其他搜索算法的比较}

\begin{center}
\begin{tabular}{|c|c|c|c|}
\hline
搜索算法 & 前提条件 & 最坏情况 & 平均情况 \\
\hline
线性搜索 & 无 & $O(n)$ & $O(n)$ \\
\hline
二分查找 & 有序数组 & $O(\log n)$ & $O(\log n)$ \\
\hline
跳跃搜索 & 有序数组 & $O(\sqrt{n})$ & $O(\sqrt{n})$ \\
\hline
插值查找 & 均匀分布有序 & $O(n)$ & $O(\log \log n)$ \\
\hline
\end{tabular}
\end{center}

\section{关键特性}

\subsection{优点}
\begin{itemize}
  \item 时间复杂度为 $O(\log n)$,对大规模数据极为高效
  \item 实现简单,容易理解
  \item 适用于任何有序数组(无需特殊分布)
  \item 稳定可靠,无边界情况下的性能退化
\end{itemize}

\subsection{缺点}
\begin{itemize}
  \item 要求输入数据\textbf{有序}
  \item 对于链表等数据结构不适用(需随机访问)
  \item 如果数据频繁修改,维护有序性的成本可能较高
\end{itemize}

\subsection{应用场景}
\begin{itemize}
  \item 数据库索引查询
  \item 系统库中的 \texttt{bsearch()} 函数
  \item 大型有序数据集的查找
  \item 编译器符号表查找
\end{itemize}

\section{总结}

二分查找通过每次消除搜索空间的一半,将查找复杂度从线性的 $O(n)$ 降低到对数的 $O(\log n)$。它是计算机科学中最基础也最重要的算法之一。

\paragraph{关键洞察}
分治思想(Divide and Conquer)的应用:通过比较一个关键元素,问题规模减半,最终达到指数级的性能改进。

\end{document}
