\documentclass[UTF8]{ctexart}
\usepackage[margin=1in]{geometry}
\usepackage{amsmath}
\usepackage{amssymb}
\usepackage{listings}
\usepackage{xcolor}
\usepackage{graphicx}
\usepackage{hyperref}

% 代码格式设置
\lstset{
    language=C++,
    basicstyle=\ttfamily\small,
    breaklines=true,
    numbers=left,
    numberstyle=\tiny,
    keywordstyle=\color{blue},
    commentstyle=\color{gray},
    stringstyle=\color{red},
    showstringspaces=false,
    backgroundcolor=\color{lightgray!20},
    frame=single
}

\title{\textbf{插值查找 (Interpolation Search)}}
\author{}
\date{}

\begin{document}

\maketitle

\section{引言:从字典查找谈起}
当我们从字典中查找单词"algorithm"时,我们不会像二分查找那样从中间开始。相反,我们会从首字母为 a 的地方开始查找,然后根据第二个字母在字母表中的位置,找到相应的位置继续查找,重复这个过程直到找到目标单词。

这个直观的查找方式就是\textbf{插值查找}的思想来源。插值查找利用了数据的分布特性,在均匀分布的数组上可以显著提高查找效率。

\section{算法思想}
插值查找(Interpolation Search)是二分查找的改良版。它的核心思想是:利用待查找元素与数组端点元素的大小关系,估计目标元素在数组中的可能位置,从而比二分查找更快地定位目标元素。

\subsection{基本原理}
假设有一个均匀分布的有序数组 $A$,我们要查找元素 $\text{key}$。不同于二分查找每次选择中点,插值查找通过以下公式估计查找位置:

\[
\text{idx} = l + \frac{\text{key} - A[l]}{A[r] - A[l]} \times (r - l)
\]

其中:
\begin{itemize}
  \item $l$ 和 $r$ 分别代表查找范围的左右索引
  \item $\text{key}$ 代表待查找的元素
  \item 公式计算出 $\text{key}$ 占据的相对位置,并将其映射到数组索引上
\end{itemize}

\subsection{示例说明}
考虑数组 $A = [1, 2, 3, \ldots, 100]$,长度为 100,相邻元素差为 1,满足均匀分布。

要查找元素 70,计算期望索引:
\[
p = \frac{70 - 1}{100 - 1} = \frac{69}{99} \approx 0.697
\]

期望索引为:
\[
\text{idx} = 0 + 0.697 \times 99 = 69
\]

对应的元素为 $A[69] = 70$,恰好就是我们要找的元素!原本用二分法需要查找 7 次,插值查找只用 1 次。

\section{算法流程}
插值查找的查找过程如下:
\begin{enumerate}
  \item 初始化 $l = 0$,$r = n-1$
  \item 计算插值位置 $\text{idx}$,并确保 $l \leq \text{idx} \leq r$
  \item 若 $A[\text{idx}] = \text{key}$,返回 $\text{idx}$(查找成功)
  \item 若 $A[\text{idx}] < \text{key}$,令 $l = \text{idx} + 1$,重复步骤 2
  \item 若 $A[\text{idx}] > \text{key}$,令 $r = \text{idx} - 1$,重复步骤 2
  \item 若 $l > r$,返回 $-1$(查找失败)
\end{enumerate}

\section{C++ 实现}

\subsection{辅助函数}
\begin{lstlisting}
// 计算插值查找的预期位置
int formula(int l, int r, int key, const int* array) {
    if (array[r] == array[l]) {
        return l;  // 避免除以零
    }
    double p = (double)(key - array[l]) / (array[r] - array[l]);
    int n = r - l;
    int idx = (int)n * p;
    return idx;
}
\end{lstlisting}

\subsection{主函数}
\begin{lstlisting}
int interpolationSearch(const int* array, int size, int key) {
    int l = 0;
    int r = size - 1;
    
    while (l <= r) {
        // 防止 idx 越界
        int x = l + formula(l, r, key, array);
        x = max(l, min(x, r));
        
        if (array[x] == key) {
            return x;
        } else if (array[x] < key) {
            l = x + 1;
        } else {
            r = x - 1;
        }
    }
    
    return -1;  // 查找失败
}
\end{lstlisting}

\subsection{完整示例程序}
\begin{lstlisting}
#include <iostream>
#include <algorithm>
using namespace std;

int formula(int l, int r, int key, const int* array) {
    if (array[r] == array[l]) {
        return l;
    }
    double p = (double)(key - array[l]) / (array[r] - array[l]);
    int n = r - l;
    int idx = (int)(n * p);
    return idx;
}

int interpolationSearch(const int* array, int size, int key) {
    int l = 0;
    int r = size - 1;
    
    while (l <= r) {
        int x = l + formula(l, r, key, array);
        x = max(l, min(x, r));
        
        if (array[x] == key) {
            return x;
        } else if (array[x] < key) {
            l = x + 1;
        } else {
            r = x - 1;
        }
    }
    
    return -1;
}

int main() {
    int arr[] = {1, 2, 3, 4, 5, 10, 20, 30, 40, 50};
    int size = sizeof(arr) / sizeof(arr[0]);
    int key = 30;
    
    int result = interpolationSearch(arr, size, key);
    
    if (result != -1) {
        cout << "元素 " << key << " found at index "
             << result << endl;
    } else {
        cout << "元素 " << key << " not found" << endl;
    }
    
    return 0;
}
\end{lstlisting}

\section{复杂度分析}

\subsection{平均情况}
当数组呈均匀分布时,插值查找的平均时间复杂度为 $O(\log \log n)$。这比二分查找的 $O(\log n)$ 要优越得多。证明过程相当复杂,感兴趣的读者可参考相关论文。

\subsection{最坏情况}
若数组分布不均匀,插值查找的复杂度会退化为 $O(n)$。

\subsubsection{极端例子}
考虑数组 $A = [1, 2, 3, \ldots, n-1, 10^9]$(即前 $n-1$ 个元素为 $1$ 到 $n-1$,最后一个元素为 $10^9$)。要查找元素 $10^9$:

\begin{itemize}
  \item 第1轮:$p = \frac{10^9 - 1}{10^9 - 1} \approx 1$,索引接近 $n-1$,获得元素,但未找到。
  \item 后续轮次:搜索范围逐渐缩小,但每次都接近右端,导致需要多次迭代。
\end{itemize}

对于含有 $n$ 个元素的这样的数组,可能需要 $O(n)$ 次比较。

\subsection{适用场景}
\begin{itemize}
  \item \textbf{适用}:数据分布均匀的有序数组,如对数均匀分布的数据
  \item \textbf{不适用}:分布不均匀的数组,此时二分查找更为稳定
\end{itemize}

\section{与二分查找的比较}
\begin{center}
\begin{tabular}{|c|c|c|}
\hline
特性 & 二分查找 & 插值查找 \\
\hline
前提条件 & 有序数组 & 均匀分布的有序数组 \\
\hline
平均复杂度 & $O(\log n)$ & $O(\log \log n)$ \\
\hline
最坏复杂度 & $O(\log n)$ & $O(n)$ \\
\hline
稳定性 & 高 & 取决于分布 \\
\hline
实现复杂度 & 低 & 中等 \\
\hline
\end{tabular}
\end{center}

\section{总结}
插值查找通过估计目标元素的位置,在均匀分布数据上提供了比二分查找更优的性能。然而,它对数据分布的要求较高,当分布不均匀时,性能会显著下降。在实际应用中,应根据数据特性选择合适的查找算法:
\begin{itemize}
  \item 数据均匀分布 $\Rightarrow$ 使用插值查找
  \item 数据分布未知或不均匀 $\Rightarrow$ 使用二分查找
\end{itemize}

\end{document}
